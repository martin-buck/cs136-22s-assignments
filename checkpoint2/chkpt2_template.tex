\documentclass[12pt]{article}
\usepackage{fullpage} 
\usepackage{microtype}      % microtypography
\usepackage{array}
\usepackage{amsmath,amssymb,amsfonts}
\usepackage{amsthm}

%% Header
\usepackage{fancyhdr}
\fancyhf{}
\fancyhead[C]{CS 136 - 2022s - Checkpoint2 Submission}
\fancyfoot[C]{\thepage} % page number
\renewcommand\headrulewidth{0pt}
\pagestyle{fancy}

\usepackage[headsep=0.5cm,headheight=2cm]{geometry}

%% Hyperlinks always blue, no weird boxes
\usepackage[hyphens]{url}
\usepackage[colorlinks=true,allcolors=black,pdfborder={0 0 0}]{hyperref}

%%% Doc layout
\usepackage{parskip}
\usepackage{times}

%%% Write out problem statements in blue, solutions in black
\usepackage{color}
\newcommand{\officialdirections}[1]{{\color{blue} #1}}

%%% Avoid automatic section numbers (we'll provide our own)
\setcounter{secnumdepth}{0}

\begin{document}
~~\\ %% add vert space

{\Large{\bf Student Names: TODO}}


{\Large{\bf Collaboration Statement:}}

Turning in this assignment indicates you have abided by the course Collaboration Policy:

\url{www.cs.tufts.edu/comp/136/2022s/index.html#collaboration-policy}

Total hours spent: TODO

We consulted the following resources:
\begin{itemize}
\item TODO
\item TODO
\item $\ldots$	
\end{itemize}

\newpage

These are the official instructions for checkpoint 2.  You can find instructions on how to submit at \url{www.cs.tufts.edu/comp/136/2022s/checkpoint2.html}

\textbf{Please consult the full project description at \url{https://www.cs.tufts.edu/comp/136/2022s/project.html} in addition to this document when working on this checkpoint.  It gives details on what we expect.}

\section{Applying Model to Dataset}

In this section, you should describe the implementation of your model from checkpoint 1.  If you have chosen to implement a model that we did not cover in a Coding Practice (CP), please describe any design choices you made when implementing it.  If you used code from a CP, please specify exactly which part of the code.  Your implementation should be your own (do not just use an existing package), but if you build off of a CP code base, you can use the implementation you submitted for the assignment, including the starter code.  If you are using a model not covered by a CP, but the starter code for a CP is useful for you, you are welcome to use it, but please describe how you have done so.  

Be sure to describe any issues you ran into when applying your model/learning method to your dataset, as well as how you have addressed them.  For example, did you have trouble scaling the model to run in a reasonable amount of time on your dataset?  If so, what changes to either the code or dataset did you make and what was the outcome?

Please additionally submit the code for this assignment in the separate Checkpoint 2 code submission.

\textbf{Section grading rubric:}
\begin{itemize}
	\item Describe how you have implemented your model (3 points)
	\item Describe any bottlenecks you ran into (3 points)
	\item Describe how you addressed bottlenecks (3 points)
	\item Submitted code to implement the model and generate results in the following section (10 points)
\end{itemize}

\section{Evaluating Hypotheses from Checkpoint 1}

In this section, you should describe the outcome of 3 of your hypotheses from checkpoint 1.  Separately for each of your 3 hypotheses, please include the information described in the example hypothesis section below.

\subsection{Hypothesis 1 (Example)}

Each hypothesis should include no more than 1/2 page of text (excluding your result).  

\begin{itemize}
 \item Briefly reiterate your hypothesis.  
 \item Describe how you evaluated your hypothesis in 2-3 sentences.  Be sure to specify your performance metric and any design choices.  For example, if you computed likelihood, be sure to specify whether it is computed on a held-out test set, and if it is, how the test set was held-out (e.g. random sampling, instances with a specific property, etc.)
 \item Include a specific result generated by the code you submit, relating to the performance metric.  This can be a graph or a table of numbers.  Your graph should include a title, a legend (where applicable), and clear labels on the axes.   
 \item Describe the behavior of the result in 1-2 sentences.  This should include a description of what you see on the graph (for example, line A is higher than line B in the left half of the graph).
 \item Analyze the implications of the result in approximately 1 paragraph.  This should link back to the specific dataset and model/learning method properties you included in your original hypothesis.  
 \item Was your hypothesis correct?  Spend 2-3 sentences reflecting on why that might be the case.
\end{itemize}

\textbf{Subsection grading rubric (for each hypothesis):}
\begin{itemize}
	\item Describe implementation details of how you evaluated your hypothesis (1 point)
	\item Include a specific result linked to the evaluation of your hypothesis (3 points)
	\item Is your result coherently presented (axis labels, titles, legends etc) (1 point)
	\item Description of the behavior of result (2 points)
	\item Analysis of implication of result (3 points)
	\item Link back to hypothesis: why was or wasn't it right? (2 points)
\end{itemize}

\section{Proposing an Upgrade to your Model or Learning Method}

This section should be no more than 1/2 page total.

Based on your results from the previous section, describe an idea for an upgrade to your model or learning method.  First, spend 2-3 sentences describing the upgrade.   You should be specific about which of the 4 options your upgrade falls into (see the project description for the full list).  Then, write a short list (3-4 elements at most) of the changes you will need to make to implement your upgrade (these don't need to be exhaustive, we just want to get you thinking about what needs to happen, and provide feedback on how to approach it).  Finally, explain why this upgrade might address a problem found in your previous hypotheses (2-3 sentences).  If you would prefer to focus on a different problem, that is also ok, but make sure to describe the problem.  

This section is largely to help us provide you with feedback and resources since you will have to include a more specific description of the upgrade in the next project checkpoint.  The more detail you include here, the more helpful our feedback will be.  While you are only required to submit one idea, you can submit up to 3 for feedback.  

\textbf{Subsection grading rubric:}
\begin{itemize}
	\item Which option does your upgrade fall under (1 point)
	\item Briefly describe your proposed upgrade (3 points)
	\item Short list of implementation changes that need to be made for upgrade (2 points)
	\item Explain why upgrade might be helpful (3 points)
\end{itemize}

\end{document}


